 %美赛模板:正文部分

\documentclass[12pt]{article}  % 官方要求字号不小于 12 号,此处选择 12 号字体
% \linespread{1.1}
% \bibliographystyle{plain}
% 本模板不需要填写年份,以当前电脑时间自动生成
% 请在以下的方括号中填写队伍控制号
\usepackage[2605876]{easymcm}  % 载入 EasyMCM 模板文件
\problem{A}  % 请在此处填写题号
%\usepackage{mathptmx}  % 这是 Times 字体,中规中矩 
\usepackage{palatino}  % mathpazo 这palatino是 COMAP 官方杂志采用的更好看的 Palatino 字体,可替代以上的 mathptmx 宏包
\usepackage{pdfpages}
\usepackage{longtable}
\usepackage{tabu}
\usepackage{threeparttable}
\usepackage{ragged2e}
\usepackage{listings}
\usepackage{paralist}
\usepackage[linesnumbered,ruled,vlined]{algorithm2e}
\graphicspath{{img/}} % 此处{img/}为相对路径,注意加上"/"
 \let\itemize\compactitem
 \let\enditemize\endcompactitem

\newcommand{\upcite}[1]{\textsuperscript{\textsuperscript{\cite{#1}}}}
\title{The Smartphone Battery:Pulse or Pause?}  % 标题

% 如需要修改题头(默认为 MCM/ICM),请使用以下命令(此处修改为 MCM)
%\renewcommand{\contest}{MCM}

\makeatletter
\newcommand{\rmnum}[1]{\romannumeral #1}
\newcommand{\Rmnum}[1]{\expandafter\@slowromancap\romannumeral #1@}
\makeatother

 %文档开始
\begin{document}

% 此处填写摘要内容
\begin{abstract}
    
The pulsing current within a smartphone battery serves as the "digital lifeline" of the modern era. However, the difficulty in predicting the non-linear variations of battery power often brings people to a pause at inappropriate moments. To reveal the secrets of the electrochemical "black box" of smartphone batteries, we established a continuous-time framework based on a hybrid model to accurately predict the State of Charge (SOC) and Time-to-Empty (TTE).

Multiple models are established: \textbf{Model I: Hybrid Kinetic-Equivalent Model}; \textbf{Mo-}\textbf{del II: SOC Estimation Model based on FFRLS-UKF}; and  \textbf{Battery Aging Model}, etc.

Before establishing the models, we collected datasets of different battery types and utilized polynomial regression to fit the \textbf{Open Circuit Voltage (OCV) curve}, laying the foundation for subsequent state estimation.

For Model I, we integrated the Equivalent Circuit Model (\textbf{ECM}) and the Kinetic Battery Model (\textbf{KiBaM}), replacing the total charge with the charge in the ``available well'' to calculate the open circuit voltage. This model can effectively predict the recovery effect and rate-capacity effect as well as calculate the terminal voltage, and determine the remaining usage time via the terminal voltage threshold.

For Model II, the Forgetting Factor Recursive Least Squares (\textbf{FFRLS}) is used to identify circuit parameters online through a series of known measurement data, while the Unscented Kalman Filter (\textbf{UKF}) corrects the SOC trajectory in real-time. Benefiting from the fitting of circuit parameters in Model II, replacing the constant circuit parameters set in Model I with these results further corrects the prediction deviation of Model I under low battery conditions. This combination achieves a \textbf{trade-off} between model complexity and reliability.

To simulate real usage scenarios, we employed Discrete-Time Markov Chains\\(\textbf{DTMC}) to generate random load current distribution curves for \textbf{five} typical user personas. Simultaneously, an aging mechanism based on the \textbf{Arrhenius equation} was integrated to quantify the capacity degradation of the battery when deviating from the optimal operating temperature. In the remaining usage time heatmaps for different personas and initial capacities, taking the Heavy Gamer as an example, their remaining usage time is reduced by up to 25\% under extreme temperatures, and the degradation increases to 45.8\% under battery aging conditions. In conclusion, \textbf{high-current voltage collapse} is the primary driver of rapid power depletion, while temperature acts as a \textbf{non-linear amplifier} of this phenomenon.

Furthermore, we performed a sensitivity analysis on the model parameters and identified the vicinity of $5^\circ C$ as the phase transition point between diffusion-limited dominance and total capacity-limited dominance. In Monte Carlo simulations addressing the uncertainties of various parameters, the simulation exhibited excellent robustness.

Finally, we synthesized our extensive findings into a visual \textbf{Smartphone Battery Care Guide} generated by Gemini, offering tailored and actionable recommendations for the five distinct user personas

    % 美赛论文中无需注明关键字。若您一定要使用,
    % 请将以下两行的注释号 '%' 去除,以使其生效
    \vspace{5pt}  %mm	毫米	1 mm = 2.845 pt   pt 点	1 pt = 0.351 mm
    \textbf{Keywords}:\textbf{lithium-ion battery,SOC estimation,KiBaM,ECM, FFRLS-UKF,DTMC}  

\end{abstract}

\maketitle  % 生成 Summary Sheet

\tableofcontents  % 生成目录


% 正文开始
% Chapter 1: Introduction
\section{Introduction}

\subsection{Problem Background}

In the era of the Internet of Everything (IoE), the smartphone has transcended its role as a mere communication tool, evolving into a "digital prosthesis" for human intelligence. However, a fundamental paradox persists: while processor speeds increase exponentially in accordance with Moore's Law, battery technology remains constrained by the inherent limits of electrochemistry. As illustrated in \textbf{Figure \ref{fig:battery_structure}}, although the battery physically dominates the internal volume of the device, it remains the primary bottleneck restricting sustained performance.

\begin{figure}[htbp]
    \centering
    \includegraphics[width=.4\textwidth]{battery_structure.jpg}
    \caption{Internal architecture of a modern smartphone.} 
    \label{fig:battery_structure}
\end{figure}

Users often simplistically attribute rapid power depletion to "heavy usage." In reality, power consumption is a complex outcome resulting from the interaction between stochastic human behavior and non-linear electrochemical kinetics. \textbf{A battery is not a linear fuel tank; rather, it is a dynamic system exhibiting the recovery effect, thermal sensitivity, and rate-capacity phenomena (Peukert's Law \cite{peukert1897uber}).} Consequently, the simple percentage indicator often serves as a lagging metric rather than a predictive measure. To bridge the gap between user perception and electrochemical reality, it is imperative to establish a robust \textbf{continuous-time mathematical framework}. By elucidating the dynamic evolution of the \textbf{State of Charge (SOC)}, we aim to demystify the "black box" of power depletion.

\subsection{Restatement of the Problem}
The objective is to construct a \textbf{mechanism-based continuous-time model} that translates stochastic usage patterns into deterministic battery state trajectories. Specifically, the tasks are formalized as follows:

\begin{itemize}
    \item \textbf{Task 1: Dynamic Model Construction.} Establish a system of differential equations to describe the instantaneous decay rate $dSOC/dt$, integrating stochastic hardware loads with non-linear environmental constraints.
    
    \item \textbf{Task 2: Quantitative Prediction and Validation.} Simulate the temporal evolution of $SOC(t)$ to accurately calculate the \textbf{Time-to-Empty ($T_{empty}$)}, defined as the critical moment when the usable energy is fully exhausted.
    
    \item \textbf{Task 3: Differential Analysis of Usage Scenarios.} Apply the model to diverse operational conditions to quantitatively compare variations in $T_{empty}$, thereby pinpointing the specific activities responsible for rapid depletion.
    
    \item \textbf{Task 4: Sensitivity and Robustness Analysis.} Evaluate the model's stability by introducing perturbations to key parameters, verifying prediction reliability under uncertainty.
\end{itemize}

\subsection{Literature Review}

The domain of battery modeling is generally stratified into three categories: electrochemical, data-driven, and equivalent circuit models (ECMs) \cite{ReviewBatteryModel}. While electrochemical approaches offer granular physicochemical insights, their prohibitive computational complexity renders them unsuitable for continuous-time smartphone simulation; similarly, data-driven methods are limited by their ``black box'' nature and dependency on extensive training datasets. Consequently, phenomenological models emerge as the optimal compromise for balancing physical fidelity with computational tractability. Among these, the \textbf{Thevenin-based ECM} \cite{KineticImproved} is widely recognized for its robust tracking of terminal voltage transients but struggles to intrinsically depict non-linear capacity recovery. In contrast, the \textbf{Kinetic Battery Model (KiBaM)} \cite{ModelingBatteryCells, BenefitsKiBaM} elegantly captures these diffusion-driven recovery and Peukert effects via a ``two-well'' charge reservoir mechanism, yet it fails to directly output the voltage metrics necessary for system cutoff analysis. This dichotomy motivates the development of a \textbf{Hybrid Kinetic-ECM Framework} in this study, integrating the voltage precision of circuit models with the dynamic capacity evolution of kinetic theory.
\begin{figure}[htbp]  %h此处,t页顶,b页底,p独立一页,浮动体出现的位置
\centering  %图表居中
\includegraphics[width=.85\textwidth]{LiteratureReview.pdf} %图片的名称或者路径之中有空格会出问题 
\caption{Literature Review Framework} % 图片标题 
\end{figure}



\subsection{Our Work}

In this paper, we conduct a systematic study to elucidate the depletion mechanisms of smartphone batteries, accurately estimate the State of Charge (SOC), and predict the remaining Time-to-Empty under realistic operating conditions. The primary contributions are organized into three progressive stages:

\begin{itemize}
    \item \textbf{Stage 1: Data Acquisition and Preprocessing.}
    We initially select high-precision battery datasets to characterize authentic battery behavior under fluctuating load profiles. Subsequently, data preprocessing techniques are employed to resample irregular timestamps into a uniform frequency. Furthermore, we establish the Open Circuit Voltage (OCV) curve via polynomial regression fitting to provide a robust baseline for state estimation.

    \item \textbf{Stage 2: Model Establishment and Adaptive Estimation.}
    We develop a \textbf{Hybrid Kinetic-Equivalent Model} to describe the continuous-time evolution of battery states. To address the non-linear recovery effect and transient voltage drops, we integrate the Kinetic Battery Model (KiBaM) with the Equivalent Circuit Model (ECM), thereby simultaneously capturing electrochemical diffusion and electrical dynamics. Recognizing that internal resistance varies with time and environment, we design an adaptive dual-observer framework: \textbf{Forgetting Factor Recursive Least Squares (FFRLS)} is utilized for online parameter identification, followed by the \textbf{Unscented Kalman Filter (UKF)} for high-precision SOC trajectory estimation.

    \item \textbf{Stage 3: Model Application and Sensitivity Analysis.}
    To evaluate battery performance across diverse usage scenarios, we apply the model to simulate five typical user personas under varying ambient temperatures. Based on the established terminal voltage dynamics, we calculate the theoretical \textbf{Time-to-Empty ($T_{empty}$)} by determining the critical moment when voltage breaches the cutoff threshold. To assess the impact of degradation, we compare runtime heatmaps between fresh and aged batteries. Finally, we conduct a comprehensive \textbf{Sensitivity Analysis} on key parameters, such as ambient temperature and aging factors, verifying the robustness and stability of our model against parametric perturbations.
\end{itemize}

\begin{figure}[htbp]  %h此处,t页顶,b页底,p独立一页,浮动体出现的位置
\centering  %图表居中
\includegraphics[width=1\textwidth]{OurWork.pdf} %图片的名称或者路径之中有空格会出问题 
\caption{Flow Chart of Our Work} % 图片标题
\label{fig:ourwork} 
\end{figure}
\vspace{-0.8cm}

\section{Model Preparation}

\subsection{Assumptions and Justifications}

To simplify the complex electrochemical processes into a computable mathematical model, we adopt the following assumptions:

\begin{itemize}
    \item \textbf{Assumption 1: During a single discharge event, the battery's maximum capacity and internal aging parameters remain constant.}
    \\ \textbf{Justification:} The timescale of battery aging is measured in months, which is significantly larger than the duration of a single discharge event.

    \item \textbf{Assumption 2: The battery is treated as a homogeneous body with uniform internal temperature.}
    \\ \textbf{Justification:} Smartphone batteries are thin and possess high thermal conductivity; thus, internal temperature gradients are negligible.

    \item \textbf{Assumption 3: Battery charge is distributed between an ''available well'' and a ''bound well'', where fluid flow represents the diffusion process.}
    \\ \textbf{Justification:} This structure is essential for mathematically reproducing the observed \textbf{recovery effect} and \textbf{rate-capacity effect} (Peukert's Law), where high-current discharge reduces apparent capacity, but rest periods allow for recovery.

    \item \textbf{Assumption 4: The device shuts down when the terminal voltage $V(t)$ drops below a critical threshold, regardless of the remaining chemical charge.}
    \\ \textbf{Justification:} Modern Power Management Integrated Circuits (PMICs) enforce a strict voltage floor to prevent undervoltage damage to the Li-ion battery.
\end{itemize}

\subsection{Notations}
The principal symbols and notations used throughout this paper are outlined in Table \ref{tb:notation}.

\begin{table}[htbp]
\centering
\caption{Nomenclature}
\renewcommand{\arraystretch}{1.2} % 自动增加行高,替代手动的 \vspace
\begin{tabular}{l l}
\toprule[1.5pt]
\textbf{Symbol} & \textbf{Description and Unit} \\
\midrule
$SOC(t)$ & State of Charge of the battery at time $t$ [\%] \\
$C_{max}$ & Maximum total charge capacity [C] \\
$V_{term}(t)$ & Terminal voltage of the battery system [V] \\
$I_{load}(t)$ & Instantaneous load current consumed by the smartphone [A] \\
$y_1(t), y_2(t)$ & Charge stored in the ``available well'' and ``bound well'' [C] \\
$V_{OC}(SOC)$ & Open Circuit Voltage, a non-linear function of SOC [V] \\
$V_{c1,2}(t)$ & Transient polarization voltages across the RC networks [V] \\
$R_0(T)$ & Temperature-dependent ohmic internal resistance [$\Omega$] \\
$R_{p1,2}, C_{p1,2}$ & Polarization resistance [$\Omega$] and polarization capacitance [F] \\
$k, c$ & Valve conductance and well-width ratio in the KiBaM model \\
$t_{off}$ & Critical time when terminal voltage drops below $V_{off}$ (Depletion time) [h] \\
\bottomrule[1.5pt]
\end{tabular}
\label{tb:notation}
\begin{tablenotes}
    \footnotesize
    \item[*] \textit{* Parameters specific to certain hardware components are defined locally in Section 4.}
\end{tablenotes}
\end{table}

\subsection{Data Collection and Visualization}

The data utilized in this study were obtained from the repositories listed in Table \ref{tab:datasource}.

\begin{table}[htbp]
\begin{center}
\caption{Summary of Data Sources and Repositories}
\label{tab:datasource}
\resizebox{\textwidth}{!}
{\begin{tabular}{c c}
\toprule[2pt]
\multicolumn{1}{m{6cm}}{\centering \textbf{Database Name}}
&\multicolumn{1}{m{10cm}}{\centering \textbf{Database URL} }\\ 
\midrule
NASA Prognostics Data Repository & https://www.nasa.gov/intelligent-systems-division/discovery/ \\
Center for Advanced Life Cycle Engineering & https://calce.umd.edu/battery-data \\
OxFord Battery Intelligence Lab & https://battery-intelligence-lab.github.io/ \\ 
Mendeley Data & https://data.mendeley.com/research-data/\\
\bottomrule[2pt]
\end{tabular}}
\end{center}
\end{table}

As illustrated in Figure \ref{fig:data}, four representative operating conditions were selected to ensure the generalization capability of the model:

\begin{itemize}
    \item \textbf{Constant Current Discharge Data (Fig. \ref{fig:CALCE}):} This dataset records the voltage response of the battery under various constant current conditions. Specifically, the low-current discharge curves were utilized to fit the nonlinear mapping relationship between the Open Circuit Voltage (OCV) and the State of Charge (SOC).
    \item \textbf{Random Walk Regime (Fig. \ref{fig:NASA}):} This profile simulates the stochastic usage patterns characteristic of daily user behavior. Due to its high degree of randomness, this dataset serves as the test set for analyzing model robustness.
    \item \textbf{HPPC Test Data (Fig. \ref{fig:MC1}):} This dataset comprises a series of vigorous charge/discharge pulses interspersed with rest periods. Such conditions effectively excite the transient polarization characteristics of the battery and are consequently employed to identify key parameters ($R_0, R_p, C_p$) within the Equivalent Circuit Model.
    \item \textbf{DST/US06 Dynamic Conditions (Fig. \ref{fig:MC2}):} These standard driving cycles involve frequent acceleration and deceleration, which corresponds to rapid switching between high and low loads in mobile device usage. This dataset is utilized to validate the prediction accuracy of the model under extreme conditions.
\end{itemize}

\begin{figure}[htbp]
    \centering
    
    % --- 子图 1 ---
    \begin{subfigure}[t]{0.42\linewidth}
        \centering
        \includegraphics[width=\linewidth]{CS2_3_9_28_11.pdf}
        \caption{Constant Current Discharge}
        \label{fig:CALCE}
    \end{subfigure}
    %\hfill % 自动撑开间距
    % --- 子图 2 ---
    \begin{subfigure}[t]{0.42\linewidth}
        \centering
        \includegraphics[width=\linewidth]{RW9_half.pdf}
        \caption{Randomized Battery Usage Data Set}
        \label{fig:NASA}
    \end{subfigure}
    
    \vspace{1em} % 行间距
    
    % --- 子图 3 ---
    \begin{subfigure}[t]{0.42\linewidth}
        \centering
        \includegraphics[width=\linewidth]{10-25-18_07.39 549_HPPC_25degC_LGHG2.pdf}
        \caption{HPPC Data Set}
        \label{fig:MC1}
    \end{subfigure}
    %\hfill
    % --- 子图 4 ---
    \begin{subfigure}[t]{0.42\linewidth}
        \centering
        \includegraphics[width=\linewidth]{A1-008-DST-US06-FUDS-25-20120827.pdf}
        \caption{DST-US06-FUDS Data Set}
        \label{fig:MC2}
    \end{subfigure}
    
    \caption{Four Different Kinds of Data Sets}
    \label{fig:data}
\end{figure}



\section{Model Establishment}

\subsection{Model \Rmnum{1}: Hybrid Kinetic-Equivalent Model}

\subsubsection{Formulation of the Hybrid Model}

To accurately predict smartphone battery autonomy under realistic, highly stochastic user patterns, relying on a singular modeling approach is often insufficient. Traditional Equivalent Circuit Models (ECM), such as the Thevenin or 2RC models, are widely employed in engineering due to their efficacy in capturing transient voltage behaviors and polarization effects through resistor-capacitor networks. However, as noted by Kasper et al. \cite{kasper2025kinetic}, standalone ECMs typically rely on simple Coulomb Counting for State of Charge (SOC) estimation. This linear approach neglects the \textbf{``recovery effect''} and \textbf{``rate dependence,''} wherein high-current discharge renders a portion of the capacity temporarily inaccessible. In contrast, the Kinetic Battery Model (KiBaM) excels at characterizing these electrochemical diffusion limitations by partitioning charge into ``bound'' and ``available'' states; yet, it lacks the circuit-level mechanisms required to describe instantaneous ohmic voltage drops and terminal voltage relaxation. Consequently, the proposal of a \textbf{Hybrid Kinetic-Equivalent Circuit Model} (as illustrated in Figure \ref{fig:KiBaM_ECM}) is both necessary and justified. By embedding the governing equations of KiBaM into the SOC estimation framework of the ECM, this hybrid methodology significantly mitigates voltage estimation errors.

\begin{figure}[htbp]
\centering
\includegraphics[width=.9\textwidth]{KiBaM_ECM.pdf}
\caption{Schematic of the Hybrid Kinetic-Equivalent Model.}
\label{fig:KiBaM_ECM}
\end{figure}

\subsubsection{Derivation of Terminal Voltage}

According to the kinetic model, charge is stored in two abstract ``wells'': the available well ($y_1(t)$) and the bound well ($y_2(t)$). Chemically, these wells correspond to the concentration of active material at the electrode surface and the deep internal chemical energy, respectively. During the discharge phase ($h_2\geq h_1$), the Open Circuit Voltage (OCV) is determined exclusively by the charge within the available well \cite{Dicke2020KiBaM}. Consequently, the global SOC and available SOC are defined as follows:
\begin{equation}
    SOC_{total}(t)=\frac{y_1(t)+y_2(t)}{c\cdot C_{max}}, \quad SOC_{T}(t)=\frac{y_1(t)}{c\cdot C_{max}}
\end{equation}

The system of differential equations governing the charge flow between the dual wells is given by:
\begin{equation} \label{eq:kibam_ode}
\left\{
\begin{aligned}
    \frac{dy_1(t)}{dt} &= -I(t) + k \cdot [h_2(t) - h_1(t)] \\
    \frac{dy_2(t)}{dt} &= -k \cdot [h_2(t) - h_1(t)]
\end{aligned}
\right.
\end{equation}
where $h_1(t) = y_1(t)/c$ and $h_2(t) = y_2(t)/(1-c)$ represent the potential heights in the respective wells.

For the equivalent circuit component (as shown in Figure \ref{fig:KiBaM_ECM}), the differential equations describing the transient circuit response can be formulated as:
\begin{equation}
    \label{eq:ecm}
    \frac{dV_{ci}(t)}{dt}=\frac{I(t)}{C_{pi}}-\frac{V_{ci}(t)}{R_{pi}C_{ci}}(i=1,2)
\end{equation}

Utilizing the functional relationship for the Open Circuit Voltage $V_{OC}$:
\begin{equation}
    \label{eq:vterm}
    V_{OC}(t) = f_{poly}(SOC_{T}(t)) = \sum_{n=0}^{6} K_n \cdot (SOC_{T}(t))^n
\end{equation}

Consequently, the expression for the terminal voltage is derived as:
\begin{equation}
    V_{term}(t)=V_{OC}(SOC_{T})-V_{c1}(t)-V_{c2}(t)-I(t)R_0
\end{equation}

In essence, provided that the OCV curve (as illustrated in Figure \ref{fig:OCV_fitting_result}) and the temporal variation of the current are known, the terminal voltage can be uniquely determined.

\begin{figure}[htbp]  %h此处,t页顶,b页底,p独立一页,浮动体出现的位置
\centering  %图表居中
\includegraphics[width=.55\textwidth]{OCV_fitting_result.pdf} %图片的名称或者路径之中有空格会出问题 
\caption{OCV Fitting Result.} % 图片标题 
\label{fig:OCV_fitting_result}
\end{figure}

\subsubsection{Estimation of Remaining Runtime}

In practical applications, device shutdown is typically triggered by the Battery Management System (BMS) when the terminal voltage drops below a safety threshold $V_{off}$ (commonly 3.0V or 3.2V) to protect the cell. Letting $t=0$ denote the initial measurement time, the remaining runtime $t_{off}$ satisfies:
\begin{equation}
    V_{term}(t_{off})=V_{off}
\end{equation}
Furthermore, according to the dual-well model, the depletion of available charge also renders the device inoperable. Consequently, the supplementary condition for charge depletion is defined as:
\begin{equation}
    SOC_T\leq SOC_{th}
\end{equation}

\subsection{Model \Rmnum{2}: SOC Estimation Based on FFRLS-UKF}

Accurate prediction of terminal voltage relies on precise circuit model parameters. However, in Model \Rmnum{1}, circuit parameters are treated as constants, which contradicts physical reality. In low SOC regions, the reduction of active electrode material induces varying degrees of parametric fluctuation \cite{chen2006accurate}. Model \Rmnum{2} utilizes the current-voltage sequence to mathematically fit instantaneous circuit parameters based on circuit equations, thereby achieving more precise SOC estimation. The flowchart for Model \Rmnum{2} is presented in Figure \ref{fig:FFRLS_UKF}.

\begin{figure}[htbp]
\centering
\includegraphics[width=.65\textwidth]{Model2.pdf}
\caption{Flow chart of FFRLS-UKF.}
\label{fig:FFRLS_UKF}
\end{figure}

\subsubsection{Parameter Update via FFRLS}

First, the Laplace transform is applied to the transient circuit equation (\ref{eq:ecm}):
\begin{equation}
    V_{ci}(s)=\frac{R_{pi}I(s)}{1+R_{pi}C_{pi}s}(i=1,2)
\end{equation}

Letting $Y(s)=V_{OC}(s)-V_{term}(s)$, we obtain:
\begin{equation}
    Y(s)=I(s)\left(R_0+\frac{R_{p1}}{1+\tau_1s}+\frac{R_{p2}}{1+\tau_2s}\right)
\end{equation}
where $\tau_1=R_{p1}C_{p1}$ and $\tau_2=R_{p2}C_{p2}$. The transfer function $G(s)$ is defined as:
\begin{equation}
    G(s)\equiv \frac{Y(s)}{I(s)}=\frac{b_0+b_1s+b_2s^2}{a_0+a_1s+a_2s^2}
\end{equation}

To transition from the frequency domain to a time-domain difference equation (where $T_s$ is the sampling period), the \textbf{Bilinear Transformation} is applied. rearranging the coefficients yields:
\begin{equation}
    Y(z^{-1})=I(z^{-1})\frac{w_3+w_4z^{-1}+w_5z^{-2}}{1-w_1z^{-1}-w_2z^{-2}}
\end{equation}
Here, $w_i(i=1,2,3,4,5)$ represents composite coefficients derived from $R, C, \text{and } T_s$. This transforms into the difference equation:
\begin{equation}
    y(k)=w_1y(k-1)+w_2y(k-2)+w_3I(k)+w_4I(k-1)+w_5I(k-2)
\end{equation}

By defining the data vector ${\phi}(k)=\left[y(k-1),y(k-2),I(k),I(k-1),I(k-2)\right]^T$ and the parameter vector to be estimated $\theta(k)=\left[w_1,w_2,w_3,w_4,w_5\right]$, the recursive process of the Forgetting Factor Recursive Least Squares (FFRLS) algorithm for online parameter updating is expressed as:
\begin{equation}
\left\{
\begin{aligned}
&y(k)=\phi^T(k)\theta(k)+\epsilon(k)\\
&K(k)=\frac{P(k-1)\phi(k)}{\lambda+\phi^T(k)P(k-1)\phi(k)}\\
&P(k)=\frac{1}{\lambda}\left[I-K(k)\phi(k)\right]P(k-1)\\
&\theta(k)=\theta(k-1)+K(k)\epsilon(k)\\
\end{aligned}
\right.
\end{equation}
where $\lambda$ is the forgetting factor (typically $0.95-0.99$), employed to adapt to the time-varying nature of circuit parameters.

\subsubsection{SOC Estimation Based on UKF}

Given the nonlinearity of $V_{OC}(SOC_k)$, the standard Kalman Filter is inapplicable; therefore, the Unscented Kalman Filter (UKF) is adopted. The system state variable is defined as $x=\left[SOC,V_{c1},V_{c2}\right]^T$, which is determined by the discrete \textbf{State Equation} $x_k=f(x_{k-1},u_{k-1})$:
\begin{equation}
    x_k=\mathbf{A}x_{k-1}+\mathbf{B}u_{k-1}=
\begin{pmatrix}
1 & 0 & 0 \\
0 & 1-\frac{T_s}{R_{p1}C_{p1}} & 0 \\
0 & 0 & 1-\frac{T_s}{R_{p2}C_{p2}}
\end{pmatrix}
x_{k-1}+
\begin{pmatrix}
    -\frac{T_s}{Q_n}\\
    \frac{T_s}{C_1}\\
    \frac{T_s}{C_2}
\end{pmatrix}u_{k-1}
\end{equation}

The \textbf{Observation Equation} is given by:
\begin{equation}
    y_k=V_{OC}(SOC_k)-V_{c1,k}-V_{c2,k}+u_kR_{0,k}
\end{equation}

\textbf{Step 1: Sigma Points Generation}
Based on the posterior state estimate $\hat{x}_{k-1}$ and the error covariance matrix $P_{k-1}$ from the previous step, a set of $2m+1$ Sigma sampling points $\chi_{k-1}$ is generated via the Unscented Transform to characterize the probability distribution of the state:
\begin{equation}
\left\{
\begin{aligned}
    \chi_{k-1}^0 &= \hat{x}_{k-1} \\
    \chi_{k-1}^i &= \hat{x}_{k-1} + \left[ \sqrt{(m+\zeta)P_{k-1}} \right]_i, & i = 1, \dots, m \\
    \chi_{k-1}^i &= \hat{x}_{k-1} - \left[ \sqrt{(m+\zeta)P_{k-1}} \right]_{i-m}, & i = m+1, \dots, 2m
\end{aligned}
\right.
\end{equation}
where $m$ is the dimension of the state vector, and $\zeta$ is a scaling parameter.

\textbf{Step 2: State Prediction}
The generated Sigma points are substituted into the discretized nonlinear state transition function $f(\cdot)$ (i.e., the battery equivalent circuit model) for time updating. The a priori state mean $\hat{x}_{k|k-1}$ is calculated via weighted summation:
\begin{equation}
\begin{aligned}
    \chi_{k|k-1}^i &= f(\chi_{k-1}^i, u_{k-1}) = \mathbf{A} \chi_{k-1}^i + \mathbf{B} u_{k-1} \\
    \hat{x}_{k|k-1} &= \sum_{i=0}^{2m} w_m^i \chi_{k|k-1}^i
\end{aligned}
\end{equation}
where $u_{k-1}$ is the load current input from the previous time step.

\textbf{Step 3: A Priori Covariance Estimation}
The prediction error covariance matrix $P_{k|k-1}$ is calculated, introducing the process noise covariance matrix $Q$ to characterize model uncertainty:
\begin{equation}
    P_{k|k-1} = \sum_{i=0}^{2m} w_c^i \left[ \chi_{k|k-1}^i - \hat{x}_{k|k-1} \right] \left[ \chi_{k|k-1}^i - \hat{x}_{k|k-1} \right]^T + Q
\end{equation}

\textbf{Step 4: Sigma Points Regeneration}
To avoid particle degeneracy and enhance estimation accuracy, a new set of Sigma points is regenerated using the updated prediction mean $\hat{x}_{k|k-1}$ and prediction covariance $P_{k|k-1}$:
\begin{equation}
\left\{
\begin{aligned}
    \chi_{k|k-1}^0 &= \hat{x}_{k|k-1} \\
    \chi_{k|k-1}^i &= \hat{x}_{k|k-1} \pm \left[ \sqrt{(m+\zeta)P_{k|k-1}} \right]_i
\end{aligned}
\right.
\end{equation}

\textbf{Step 5: Measurement Prediction}
The regenerated Sigma points are mapped to the observation space via the nonlinear observation function $g(\cdot)$ (incorporating the nonlinear OCV-SOC relationship) to calculate the predicted terminal voltage mean $\hat{y}_{k|k-1}$:
\begin{equation}
\begin{aligned}
    y_{k|k-1}^i &= g(\chi_{k|k-1}^i, u_k) = V_{OC}(SOC^i) - V_{c1}^i - V_{c2}^i - R_0 u_k \\
    \hat{y}_{k|k-1} &= \sum_{i=0}^{2m} w_m^i y_{k|k-1}^i
\end{aligned}
\end{equation}

\textbf{Step 6: Measurement Update}
The observation covariance $P_{yy}$ and the state observation cross-covariance $P_{xy}$ are calculated to solve for the Kalman Gain $K_k$. The predicted state is then corrected using the actual measured battery voltage $y_k$ to obtain the final SOC estimate:
\begin{equation}
\begin{aligned}
    P_{yy} &= \sum_{i=0}^{2m} w_c^i (y^i - \hat{y})(y^i - \hat{y})^T + R \\
    P_{xy} &= \sum_{i=0}^{2m} w_c^i (\chi^i - \hat{x})(y^i - \hat{y})^T \\
    K_k &= P_{xy} P_{yy}^{-1} \\
    \hat{x}_k &= \hat{x}_{k|k-1} + K_k (y_k - \hat{y}_{k|k-1}) \\
    P_k &= P_{k|k-1} - K_k P_{yy} K_k^T
\end{aligned}
\end{equation}
where $R$ is the measurement noise covariance matrix, representing voltage sensor error.

\subsection{Refinement of Model \Rmnum{1} via Model \Rmnum{2}}

\begin{figure}[htbp]
    \centering
    \includegraphics[width=1\textwidth]{Battery_Parameter_Fitting.pdf}
    \caption{Identified parameters ($V_{OC},R_0, R_{p1},R_{p2} C_{p1},C_{p2}$) varying with SOC.}
    \label{fig:parameter_change}
\end{figure}

In the preliminary construction of Model \Rmnum{1}, the ohmic internal resistance $R_0$ and polarization impedances $R_{p1,2}, C_{p1,2}$ were approximated as constants to simplify computational complexity. However, based on the online identification results derived from the FFRLS algorithm in Model \Rmnum{2} under HPPC dynamic conditions (as shown in Figure \ref{fig:parameter_change}), significant time-varying characteristics of the circuit parameters were observed, particularly within the low SOC region.

An analysis of the identification results reveals that the ohmic internal resistance $R_0$ exhibits an exponential growth trend as SOC decreases. This phenomenon elucidates why mobile devices are prone to overheating and sudden voltage drops during low-battery states.
To enhance the physical fidelity of Model \Rmnum{1}, the discrete parameters identified by FFRLS were fitted into continuous functions with respect to SOC, yielding $R_0(SOC)$, $R_{p1}(SOC)$, $R_{p2}(SOC)$, $C_{p1}(SOC)$, and $C_{p2}(SOC)$. The specific expression for $R_0(SOC)$ is derived as follows:
\begin{equation}
    R_0(SOC) = 0.16e^{-24.37SOC}+0.07(\Omega)
\end{equation}

Subsequently, the fitted parameter function $R_0(SOC)$ (along with the others) was substituted back into the kinetic equations of the hybrid model. Figure \ref{fig:Compare} presents a comparative analysis involving the standard ECM, the original Model \Rmnum{1}, Model \Rmnum{2}, and the modified Model \Rmnum{1}:
\begin{itemize}
    \item \textbf{Terminal Voltage Prediction (Fig. b):} During the end-of-discharge phase, the constant-parameter model yields voltage predictions significantly higher than the actual values due to the neglect of internal resistance surges. In contrast, the modified Model \Rmnum{1} substantially improves the prediction accuracy of the remaining runtime.
    \item \textbf{SOC Estimation Error (Fig. c):} The open-loop cumulative error of Model \Rmnum{1} tends to drift over time. Conversely, Model \Rmnum{2}, benefiting from parameter adaptation and UKF feedback correction, consistently maintains the error within $\pm 0.5\%$, demonstrating exceptional robustness.
\end{itemize}
\begin{figure}[htbp]
    \centering
    \includegraphics[width=0.85\textwidth]{Battery_Simulation_Comparison_with_Zoom_Fixed.pdf}
    \caption{Performance comparison: (a) SOC trajectories, (b) Terminal voltage response, (c) Estimation error (d) Voltage error.}
    \label{fig:Compare}
\end{figure}

In summary, Model \Rmnum{1}, following dynamic parameter correction, demonstrates significantly enhanced predictive precision.



\section{Modeling Applications and Problem Solving}

\subsection{Analysis of Model Application Scenarios}

\subsubsection{Power Consumption Classification of Smartphone Components}

We categorize the primary power consumption sources of a smartphone into four components: Screen, CPU, GPS, and WIFI. The power consumption of each component correlates with specific variables, as illustrated in Fig. \ref{fig:Power}:

\begin{figure}[htbp]
    \centering
    % 保留你原有的设置:减小标题与图片的垂直间距
    \setlength{\abovecaptionskip}{2pt}
    
    % --- 第一行 ---
    \begin{subfigure}[t]{0.48\linewidth}
        \centering
        \includegraphics[width=\linewidth]{SCREEN_Power_3D_Model.pdf}
        \caption{OLED Screen Power Consumption.}
        \label{fig:Screen}
    \end{subfigure}
    \hfill % 自动填充中间的空白,使两图对齐到左右两端
    \begin{subfigure}[t]{0.48\linewidth}
        \centering
        \includegraphics[width=\linewidth]{CPU_Power_Time_Effect.pdf} 
        \caption{CPU Power consumption.}
        \label{fig:CPU}
    \end{subfigure}
    
    % --- 行间距 ---
    \vspace{2pt}
    
    % --- 第二行 ---
    \begin{subfigure}[t]{0.48\linewidth}
        \centering
        \includegraphics[width=\linewidth]{WIFI_Power_Model_3D.pdf}
        \caption{WIFI Power consumption}
        \label{fig:WIFI}
    \end{subfigure}
    \hfill
    \begin{subfigure}[t]{0.48\linewidth}
        \centering
        \includegraphics[width=\linewidth]{GPS_Power_Model_Calibrated_2D.pdf}
        \caption{GPS Component Power Model.}
        \label{fig:GPS}
    \end{subfigure}
    
    \caption{Four Different Power Consumption Models of Smartphone Components.}
    \label{fig:Power}
\end{figure}



\begin{itemize}
    \item \textbf{Screen:} The power consumption is primarily determined by brightness ($B$) and the Average Picture Level ($APL$):
    \begin{equation}
        P_{screen}=P_{base}+P_{peak}\cdot \left(B\cdot APL\right)^\gamma
    \end{equation}
    
    \item \textbf{CPU:} CPU power depends on utilization, temperature, and operating frequency. The calculation formulas differ between Efficiency Cores and Performance Cores:
    \begin{equation}
        \left\{
        \begin{aligned}
            &P_{dyn}(u)=uC_EV^2_E(u)f_E(u)\quad 0\leq u \leq u_{th}  \\
            &P_{static}(u,T)=K_EV_E(u)e^{\alpha(T-T_0)}\\
        \end{aligned}
        \right.(\text{Efficiency Core})
    \end{equation}
    \begin{equation}
        \left\{
        \begin{aligned}
            &P_{dyn}(u)=u(C_E+C_P)V^2_P(u)f_P(u)\quad u_{th}< u \leq 1  \\
            &P_{static}(u,T)=(K_PV_P(u)+K_EV_{Emin})e^{\alpha(T-T_0)}\\
        \end{aligned}
        \right.(\text{Performance Core})
    \end{equation}
    The total CPU power is the sum of static and dynamic power.
    
    \item \textbf{WIFI:} WIFI power consumption is mainly governed by the transmission rate ($R_{reg}$) and signal strength ($RSSI$). The model exhibits a plateau (see Fig. \ref{fig:WIFI}), indicating that when the signal is weak, transmission power cannot increase indefinitely to maintain connectivity.
    \begin{equation}
        \left\{
        \begin{aligned}
            &C(RSSI)=\gamma\cdot\frac{C_{max}}{1+e^{-k(RSSI-RSSI_0)}}   \\
            &P_{WIFI}=P_{static}+\frac{R_{reg}}{C(RSSI)}(P_{max}-P_{static})\\
        \end{aligned}
        \right.
    \end{equation}
    
    \item \textbf{GPS:} According to related literature \cite{tawalbeh2016greener}, considering only the Signal-to-Noise Ratio ($SNR$) yields a power estimation accuracy of up to 91\%. Therefore, we simplify this as a linear model of $SNR$.
    \begin{equation}
        P=P_{static}+k\cdot SNR
    \end{equation}
\end{itemize}


The total power of the smartphone can be approximated as the sum of these four main components, while the total load current is the ratio of total power to the terminal voltage:
\begin{equation}
    P_{total}=P_{screen}+P_{CPU}+P_{WIFI}+P_{GPS}
\end{equation}
\begin{equation}
    I_{load}=\frac{P_{total}}{V_{term}}
\end{equation}

\subsubsection{Current Distribution Analysis}

Daily smartphone usage encompasses multiple scenarios, as illustrated in Fig. \ref{fig:DailyCycle}. To precisely analyze individual usage patterns and duration, we categorize usage scenarios into five distinct classes: Idle, Gaming, Navigation, Video Streaming, and Web Browsing. Since component power consumption varies across these scenarios, the resulting current distribution is presented in Fig. \ref{fig:Senario}.

\begin{figure}[htbp]
    \centering
    \begin{minipage}[b]{0.45\textwidth}
        \centering
        \includegraphics[width=\textwidth]{Daily Cycle of Smartphone Using.pdf}
        \captionof{figure}{Daily Cycle of Smartphone Usage.}
        \label{fig:DailyCycle}
    \end{minipage}
    \hfill
    \begin{minipage}[b]{0.45\textwidth}
        \centering
        \includegraphics[width=\textwidth]{Figure_Scenario_Power_Analysis.pdf}
        \captionof{figure}{Power Analysis in Different Scenarios.}
        \label{fig:Senario}
    \end{minipage}
\end{figure}

Subsequently, we classify user groups with distinct usage habits into five representative personas: Heavy Gamer, Office Worker, Driver, Video Streamer, and Minimalist.

To construct a dynamic load model that reflects real-world conditions, we employ a Discrete-time Markov Chain (DTMC) to simulate user behavior trajectories over the course of a day. We define the smartphone's operating states as the state space $\mathbf{S}$. Let $X_t$ denote the smartphone state at time $t$. The user's behavioral pattern is described by the state transition probability matrix $\mathbf{P}$, where the element $P_{ij}$ represents the probability of the user transitioning from current state $i$ to the next state $j$.

Distinct user personas are determined by the specific characteristics of their transition matrices. Taking the \textbf{Heavy Gamer} as an example, this persona exhibits a high degree of "stickiness" to the gaming state. The transition matrix $\mathbf{P}_{gamer}$ defined in our model is as follows:

\begin{equation}
\mathbf{P}_{gamer} = 
\bordermatrix{
 & \text{Idle} & \text{Web} & \text{Video} & \text{Game} & \text{Navi} \cr
\text{Idle} & 0.90 & 0.04 & 0.01 & 0.05 & 0.00 \cr
\text{Web} & 0.20 & 0.70 & 0.05 & 0.05 & 0.00 \cr
\text{Video} & 0.05 & 0.05 & 0.85 & 0.05 & 0.00 \cr
\text{Game} & 0.02 & 0.00 & 0.01 & \mathbf{0.97} & 0.00 \cr
\text{Navi} & 0.50 & 0.10 & 0.00 & 0.00 & 0.40 
}
\end{equation}

In this matrix, the fourth row corresponds to the gaming state ($S_{game}$). It can be observed that the self-transition probability is $P_{game \rightarrow game} = 0.97$. This implies that once the user enters a gaming scenario, there is a 97\% probability that they will remain in the gaming state for the subsequent minute. This high auto-regressive probability accurately simulates the load characteristics associated with the prolonged continuous gaming sessions typical of heavy players.

Based on the defined transition matrices, we utilized Monte Carlo Simulation to generate state sequences for a full day. These state sequences were then mapped to current time series $I(t)$, resulting in the daily current distribution for different personas shown in Fig. \ref{fig:Persona}.

\begin{figure}[h]
    \centering
    \includegraphics[width=1\textwidth]{Figure_Multi_Persona_Clean.pdf}
    \caption{Power Analysis in Different Personas.}
    \label{fig:Persona}
\end{figure}

\subsubsection{Battery Aging during Usage}

According to the review paper \cite{vermeer2022comprehensive}, lithium battery aging primarily manifests as capacity fade. This phenomenon is driven by three main factors: Loss of Lithium Inventory (LLI), Loss of Active Material (LAM), and Conductivity Loss (CL). 

The aging process is influenced by various stress factors, with temperature being a critical determinant. High temperatures accelerate the rate of side reactions (following the Arrhenius equation) and Solid Electrolyte Interphase (SEI) growth, whereas low temperatures slow down diffusion rates, thereby promoting lithium plating. We model the total capacity loss as a superposition of these two mechanisms:
\begin{equation}
    Q_{loss}=\left[A_{SEI}\exp{(\frac{-E_{SEI}}{RT})}+A_{Plating}\exp{(\frac{E_{Plating}}{RT})}\right]\cdot N^z
\end{equation}

where $N$ represents the cycle number and $z$ is the fitted power-law exponent. The variation of battery aging with temperature and cycle count is shown in Fig. \ref{fig:aging}, indicating that capacity loss is minimized at approximately $40^\circ$C.

\begin{figure}[htbp]
    \centering
    \includegraphics[width=0.8\textwidth]{aging_multi_cycle_analysis.pdf}
    \caption{Battery Aging Analysis.}
    \label{fig:aging}
\end{figure}

\subsection{Estimation of Remaining Runtime}

\subsubsection{Runtime Prediction Across Different Scenarios}
Utilizing the enhanced hybrid kinetic model, comprehensive simulations of battery autonomy were conducted for five distinct user personas across a temperature range of $-15^\circ\text{C}$ to $100^\circ\text{C}$. The results encompass both the early stage (Cycle 25, Fig. \ref{fig:cycle25}) and the mid-to-late stage (Cycle 250, Fig. \ref{fig:cycle250}) of the battery lifecycle. In the heat maps, darker hues denote longer runtime durations.

\begin{figure}[htbp]
    \centering
    % --- 子图 1 ---
    \begin{subfigure}[b]{0.48\textwidth}
        \centering
        % 宽度设为 \linewidth 即可,指当前子图环境的宽度
        \includegraphics[width=\linewidth]{cycle25.pdf}
        \caption{Battery Runtime under 25 Cycles.}
        \label{fig:cycle25}
    \end{subfigure}
    \hfill % 撑开左右间距
    % --- 子图 2 ---
    \begin{subfigure}[b]{0.48\textwidth}
        \centering
        \includegraphics[width=\linewidth]{cycle250.pdf}
        \caption{Battery Runtime under 250 Cycles}
        \label{fig:cycle250}
    \end{subfigure}
    
    % --- 总标题 ---
    \caption{Battery Runtime of Different Personas.}
    \label{fig:personas}
\end{figure}

\begin{itemize}
    \item \textbf{Impact of User Behavioral Patterns:} A horizontal analysis of the heat maps reveals that load intensity is the dominant determinant of battery autonomy. Regardless of ambient temperature variations, runtime exhibits a strictly monotonically decreasing trend across the Minimalist, Office Worker, Driver, Video Streamer, and Heavy Gamer personas. The Heavy Gamer experiences the shortest autonomy due to frequent invocation of high-power CPU and GPU resources, which sustains high load currents. Conversely, the Minimalist, characterized primarily by idle and light browsing states, maintains currents at minimal levels, thereby achieving the longest runtime.
    \item \textbf{Coupling Effect of Temperature and Aging:} Longitudinally, the heat maps reveal that extended runtimes are concentrated in the central temperature region, correlating with the previously modeled temperature-dependent capacity loss. The battery demonstrates optimal endurance at approximately $40^\circ C$, while deviations toward high or low extremes attenuate performance. A comparison between the two panels indicates that as the charge-discharge cycle count increases, the overall intensity of the heat map fades, signifying a significant degradation in global battery autonomy.
\end{itemize}

Comprehensive analysis suggests that the core driver of rapid battery depletion is \textbf{high-computation tasks} (e.g., gaming, high-definition rendering), where the voltage drop induced by high currents significantly outweighs the impact of screen-on time. Furthermore, ambient temperature acts as an amplifier; deviating from the optimal thermal window exacerbates runtime reduction.



\subsubsection{Runtime Prediction at Varying Initial Capacities}
To assess battery performance under partial charge conditions, the ``Video Streamer'' persona was selected as a representative case study to simulate autonomy across different initial State of Charge (SOC) levels. Figures \ref{fig:ini25} and \ref{fig:ini250} illustrate the predictive outcomes for the 25th and 250th cycles, respectively.

\begin{figure}[htbp]
    \centering
    % --- 子图 1 ---
    % 使用 [b] 参数让两张图底部对齐(通常能保证子标题在同一高度)
    \begin{subfigure}[b]{0.48\textwidth}
        \centering
        \includegraphics[width=\linewidth]{ini25.pdf}
        \caption{Battery Runtime under 25 Cycles (Video Streamer).}
        \label{fig:ini25}
    \end{subfigure}
    \hfill % 撑开左右间距
    % --- 子图 2 ---
    \begin{subfigure}[b]{0.48\textwidth}
        \centering
        \includegraphics[width=\linewidth]{ini250.pdf}
        \caption{Battery Runtime under 250 Cycles (Video Streamer).}
        \label{fig:ini250}
    \end{subfigure}
    
    % --- 总标题 ---
    \caption{Battery Runtime Prediction with Different Initial SOC.}
    \label{fig:Initial}
\end{figure}

\begin{itemize}
    \item \textbf{Buffering Effect at High Charge Levels:} At high initial SOC levels, the battery operates within the flat region of the Open Circuit Voltage (OCV) curve. The elevated potential effectively counteracts internal resistance fluctuations caused by extreme temperatures. Even under the aging condition of 250 cycles, distinct differences in runtime persist between ambient and extreme temperatures.
    \item \textbf{Voltage Collapse at Low Charge Levels:} When the SOC diminishes to 20\%, the operating point approaches the steep descent region of the OCV curve. Consequently, the correlation between battery autonomy and temperature becomes less pronounced. At this stage, the battery has essentially lost the majority of its power output capability.
\end{itemize}

Comparative analysis of the two figures indicates that aging not only truncates the overall runtime but also compresses the ``usable environmental window.'' While a fresh battery retains 7-9 hours of utility even at low charge, an aged battery is reduced to a mere 1-3 hours. Therefore, for aged batteries, avoiding high-intensity activities and extreme temperatures is critical to preventing unexpected shutdowns.




\section{Model Evaluation and Discussion}

\subsection{Sensitivity Analysis}

To evaluate model robustness under parameter uncertainty and identify key physical drivers, we employed local sensitivity analysis and Monte Carlo simulations by introducing a $\pm 5\%$ random perturbation to key parameters ($C_{total}, E_{plating}, E_{SEI}, k$).

\textbf{Temperature-Driven Mechanism Competition (Fig. \ref{fig:sens_trend}):} The sensitivity index ($|S|$) exhibits strong temperature dependence, revealing the evolution of aging mechanisms. In the low-temperature region ($T < 5^\circ\text{C}$), the model is most sensitive to $E_{plating}$ ($|S| \approx 0.52$), reflecting the diffusion-limited lithium plating bottleneck. As temperature rises ($T > 10^\circ\text{C}$), diffusion constraints vanish, and $C_{total}$ becomes the dominant factor. The crossover point around $5^\circ\text{C}$ marks the physical transition from a "kinetically limited" to a "capacity-limited" regime.

\textbf{Model Robustness Verification (Fig. \ref{fig:monte_carlo}):} A 1000-sample Monte Carlo simulation was conducted for the "Video Streamer" scenario at $25^\circ\text{C}$. The runtime distribution follows a Gaussian profile with a mean of 20.05h. The extremely low coefficient of variation (CV = 0.73\%) and narrow 95\% confidence interval indicate that despite physical parameter uncertainty, the model maintains high determinacy.

\begin{figure}[htbp]
    \centering
    \begin{minipage}[t]{0.48\textwidth}
        \centering
        \includegraphics[width=\textwidth]{SensitiveAnalysis.pdf}
        \caption{Smoothed Sensitivity Trends: Mechanism Competition across Temperatures.}
        \label{fig:sens_trend}
    \end{minipage}
    \hfill
    \begin{minipage}[t]{0.48\textwidth}
        \centering
        \includegraphics[width=\textwidth]{Monte_Carlo.pdf}
        \caption{Monte Carlo Determinacy Analysis: Video Streamer @ $25^\circ$C.}
        \label{fig:monte_carlo}
    \end{minipage}
\end{figure}

\textbf{Micro-Parameter Response (Fig. \ref{fig:activation_energy}):} The impact of electrochemical barriers aligns with the Arrhenius law. Lower activation energies (darker curves) for $E_{plating}$ and $E_{SEI}$ correspond to lower reaction barriers, resulting in accelerated SOC decline rates. This confirms the model's ability to correctly map microscopic kinetic parameters to macroscopic performance degradation.

\begin{figure}[htbp]
    \centering
    % --- 子图 1 ---
    \begin{subfigure}[b]{0.48\textwidth}
        \centering
        % [width=\linewidth] 指的是相对于当前 subfigure 的宽度
        \includegraphics[width=\linewidth]{SOC_E_plate_Evolution_Spectral.pdf}
        \caption{Impact of Plating Activation Energy ($E_{plate}$)}
        \label{fig:plate_energy} % 建议添加子图标签
    \end{subfigure}
    \hfill % 撑开左右间距
    % --- 子图 2 ---
    \begin{subfigure}[b]{0.48\textwidth}
        \centering
        \includegraphics[width=\linewidth]{SOC_E_SEI_Evolution_Spectral.pdf}
        \caption{Impact of SEI Activation Energy ($E_{SEI}$)}
        \label{fig:sei_energy} % 建议添加子图标签
    \end{subfigure}
    
    % --- 总标题 ---
    \caption{Sensitivity Analysis of Activation Energies on SOC Depletion Profiles.}
    \label{fig:activation_energy}
\end{figure}

\textbf{Macro-Capacity Scaling (Fig. \ref{fig:capacity_sens}):} Unlike activation energies that alter the curvature of the discharge profile, total capacity ($C_{total}$) demonstrates a linear scaling effect on the time axis. The proportional compression of runtime with capacity loss explains why $C_{total}$ dominates sensitivity in the non-extreme temperature regions observed in Fig. \ref{fig:sens_trend}.

\begin{figure}[htbp]
    \centering
    \includegraphics[width=0.5\textwidth]{battery_soc_degradation.pdf}
    \caption{SOC Discharge Profiles under Capacity Fade (Video Streamer, $25^\circ$C).}
    \label{fig:capacity_sens}
\end{figure}

\subsection{Strengths and Weaknesses}

\subsubsection{Strengths}
\begin{itemize}
    \item \textbf{High-Fidelity Hybrid Modeling Framework:} We innovatively developed a hybrid KiBaM-ECM framework. This architecture synergizes the ECM's ability to capture transient voltage (I-V response) with the KiBaM's "two-well" kinetic structure for describing non-linear capacity recovery. This effectively resolves the prediction bias inherent in traditional ECMs under intermittent load conditions.
    \item \textbf{Adaptive Robust Estimation Algorithm:} By integrating a dual FFRLS-UKF observer, we achieved online identification of time-varying parameters (e.g., internal resistance aging) and closed-loop SOC correction. This algorithm effectively mitigates current sensor noise and initial state errors, maintaining exceptional convergence and determinacy (CV=0.73\%) in long-term simulations.
    \item \textbf{Multi-Physics Coupled Aging Mechanism:} Unlike simple empirical decay formulas, our model integrates microscopic electrochemical mechanisms governed by the Arrhenius law (SEI film growth and lithium plating). This allows for a quantitative explanation of irreversible damage caused by extreme temperatures ($-15^\circ\text{C}$) and high C-rate discharges.
    \item \textbf{Stochastic User Behavior Simulation:} We utilized Discrete-time Markov Chains (DTMC) to construct dynamic load profiles that mirror real-world usage. This enables the model to assess battery endurance across diverse user personas (e.g., Heavy Gamers vs. Minimalists), thereby enhancing the model's generalization capabilities.
\end{itemize}

\subsubsection{Weaknesses}
\begin{itemize}
    \item \textbf{High Computational Complexity:} The UKF and FFRLS algorithms involve extensive matrix operations. Compared to simple Look-up Table (LUT) methods, implementing this algorithm in real-time on resource-constrained embedded systems (e.g., low-end power management ICs) poses computational challenges.
    \item \textbf{Dependency on Experimental Data:} The model's high precision relies on the quality of HPPC and OCV test data. Any change in the battery chemistry (e.g., switching from NCM to LFP) necessitates a complete re-identification of offline parameters.
    \item \textbf{Homogenized Thermal Assumption:} To simplify calculations, we assumed a uniform internal temperature distribution, neglecting thermal gradients between the cell core and surface. This simplification may underestimate local thermal runaway risks during ultra-high rate discharges.
\end{itemize}

\section{Conclusion and Recommendations}

Based on our continuous-time mathematical model, specifically analyzing the \textbf{Kinetic Battery Model (KiBaM)} dynamics and \textbf{SEI film aging mechanisms}, we provide the following scientifically backed recommendations to extend your smartphone's daily runtime and long-term lifespan.

\begin{itemize}
    % Recommendation 1
    \item \textbf{The ``Pulse \& Pause'' Usage Strategy}
    
    \textbf{Recommendation:} Take short breaks (5--10 min) during high-drain tasks (e.g., gaming) instead of continuous usage to maximize runtime.
    
    \textbf{Model Evidence:} Our \textbf{Hybrid KiBaM-ECM Model} confirms the ``Recovery Effect.'' High loads deplete the ``Available Well'' ($y_1$) rapidly, causing sharp voltage drops and premature cutoff. Pausing allows diffusion ($k$) from the ``Bound Well'' ($y_2$) to refill $y_1$, recovering terminal voltage and extending usable time by 10--15\%.

    % Recommendation 2
    \item \textbf{The ``Goldilocks Zone'' for Temperature}
    
    \textbf{Recommendation:} Avoid usage in extreme temperatures; maintain device temperature between $10^\circ\mathrm{C}$ and $35^\circ\mathrm{C}$ to prevent anomalies.
    
    \textbf{Model Evidence:} 
    \begin{itemize}
        \item \textbf{Low Temp ($<0^\circ\mathrm{C}$):} Internal resistance $R_0$ rises exponentially (Arrhenius equation), causing massive $I \times R$ voltage drops and false ``empty'' signals.
        \item \textbf{High Temp ($>35^\circ\mathrm{C}$):} Our \textbf{Aging Model} shows that heat significantly accelerates SEI growth and side reactions, permanently reducing total capacity ($C_{total}$).
    \end{itemize}

    % Recommendation 3
    \item \textbf{The 20--80\% Charging Rule}
    
    \textbf{Recommendation:} Maintain SOC between 20\% and 80\%. Avoid deep discharges and leaving the device at 100\% overnight.
    
    \textbf{Model Evidence:} Deep discharge ($<20\%$) stresses diffusion mechanics, risking voltage collapse as the gradient between wells diminishes. Conversely, high potentials ($>80\%$) induce lattice stress and lithium plating, the primary drivers of permanent capacity fade (SOH degradation).

    % Recommendation 4
    \item \textbf{Load-Aware Charging}
    
    \textbf{Recommendation:} For heavy users, prefer slow overnight charging over fast-charging during active usage to reduce thermal stress.
    
    \textbf{Model Evidence:} Our \textbf{FFRLS-UKF} analysis indicates that fast charging under load creates a ``double heating'' effect (Ohmic + Metabolic heat). Sensitivity analysis identifies this as the worst-case scenario for battery longevity.
\end{itemize}

\begin{figure}[htbp]
    \centering
    \includegraphics[width=0.6\textwidth]{Gemini_Generated_Image_n9vukin9vukin9vu.png}
    \caption{Smartphone Battery Care Guide Generated by Google Gemini}
    \label{fig:poster}
\end{figure}

% 参考文献,此处以 MLA 引用格式为例
\clearpage   %另起一页继续写。这时,你最好使用"\clearpage" 
% 指定参考文献排版风格,美赛常用 plain 或 unsrt (按引用顺序排序)
\bibliographystyle{unsrt}

% 指定你的 .bib 文件名 (不需要加 .bib 后缀)
\bibliography{reference.bib}

% \includepdf[pages={1,2}]{Memo.pdf} 
% =========================================
% Report on Use of AI (Must be included!)
% =========================================
\newpage % 强制换页,单独占一页
\section*{Report on Use of AI} % 不带序号的标题

% 1. 声明引言
This report describes the AI tools used in our modeling process, code generation, and writing assistance, in accordance with the COMAP contest policy.


This section specifically outlines the utilization of Generative AI in creating the visual recommendations presented in our work.

\begin{table}[htbp]
    \centering
    \caption{AI Usage Description for Poster Generation}
    \label{tab:ai_use_poster}
    \renewcommand{\arraystretch}{1.5}
    \begin{tabular}{|l|l|p{9cm}|}
    \hline
    \textbf{AI Tool} & \textbf{Purpose} & \textbf{Usage Details} \\ \hline
    Nano Banana & Image Generation & 
    Used to synthesize the educational infographic poster (Figure \ref{fig:poster}). \newline
    \textbf{Methodology:} We input a structured prompt derived from our model's conclusions to visualize the "Battery Health Guide." \newline
    \textbf{Prompt Used:} "An educational infographic poster titled 'SMARTPHONE BATTERY CARE GUIDE'. The poster is divided into 5 distinct, colorful horizontal sections, illustrating different user personas (Minimalist, Office Worker, Driver, Streamer, Gamer) with a modern flat vector illustration style..." \newline
    \textbf{Outcome:} The tool generated the visual representation of our customized suggestions, which was incorporated into the paper to enhance public accessibility. \\ \hline
    \end{tabular}
\end{table}
% 3. 官方要求的责任声明 (至关重要,不要修改)
\vspace{1cm}
\noindent \textbf{Assurances}

\noindent The content of this paper is entirely the work of the team members. We have verified the outputs provided by the AI tools and take full responsibility for the accuracy and integrity of the model and results presented in this report.

\end{document}  % 结束